%%%%%%%%%%%%%%%%%%%%%%%%%%%%%%%%%%%%%%%%%
% Journal Article
% LaTeX Template
% Version 1.4 (15/5/16)
%
% This template has been downloaded from:
% http://www.LaTeXTemplates.com
%
% Original author:
% Frits Wenneker (http://www.howtotex.com) with extensive modifications by
% Vel (vel@LaTeXTemplates.com)
%
% License:
% CC BY-NC-SA 3.0 (http://creativecommons.org/licenses/by-nc-sa/3.0/)
%
%%%%%%%%%%%%%%%%%%%%%%%%%%%%%%%%%%%%%%%%%

%----------------------------------------------------------------------------------------
%	PACKAGES AND OTHER DOCUMENT CONFIGURATIONS
%----------------------------------------------------------------------------------------

\documentclass[twoside,twocolumn]{article}
\usepackage[sorting=none]{biblatex}
\usepackage{longtable}
\bibliography{Bibliography.bib}
\bibliography{PlamenBibliography.bib}
\bibliography{NonPlamenBibliography.bib}
\tolerance=1
\emergencystretch=\maxdimen
\hyphenpenalty=10000
\hbadness=10000
\usepackage{graphicx}
\usepackage{float}
\usepackage{csquotes}
\usepackage{blindtext} % Package to generate dummy text throughout this template

\usepackage[sc]{mathpazo} % Use the Palatino font
\usepackage[T1]{fontenc} % Use 8-bit encoding that has 256 glyphs
\linespread{1.05} % Line spacing - Palatino needs more space between lines
\usepackage{microtype} % Slightly tweak font spacing for aesthetics

\usepackage[english]{babel} % Language hyphenation and typographical rules

\usepackage[hmarginratio=1:1,top=32mm,columnsep=20pt]{geometry} % Document margins
\usepackage[hang, small,labelfont=bf,up,textfont=it,up]{caption} % Custom captions under/above floats in tables or figures
\captionsetup[table]{name=Tabla}
\usepackage{booktabs} % Horizontal rules in tables

\usepackage{lettrine} % The lettrine is the first enlarged letter at the beginning of the text

\usepackage{enumitem} % Customized lists
\setlist[itemize]{noitemsep} % Make itemize lists more compact

\usepackage{abstract} % Allows abstract customization
\renewcommand{\abstractnamefont}{\normalfont\bfseries} % Set the "Abstract" text to bold
\renewcommand{\abstracttextfont}{\normalfont\small\itshape} % Set the abstract itself to small italic text

\usepackage{titlesec} % Allows customization of titles
\renewcommand\thesection{\Roman{section}} % Roman numerals for the sections
\renewcommand\thesubsection{\roman{subsection}} % roman numerals for subsections
\titleformat{\section}[block]{\large\scshape\centering}{\thesection.}{1em}{} % Change the look of the section titles
\titleformat{\subsection}[block]{\large}{\thesubsection.}{1em}{} % Change the look of the section titles

\usepackage{fancyhdr} % Headers and footers
\pagestyle{fancy} % All pages have headers and footers
\fancyhead{} % Blank out the default header
\fancyfoot{} % Blank out the default footer
%\fancyhead[C]{Running title $\bullet$ May 2016 $\bullet$ Vol. XXI, No. 1} % Custom header text
\fancyfoot[RO,LE]{\thepage} % Custom footer text

\usepackage{titling} % Customizing the title section

\usepackage{hyperref} % For hyperlinks in the PDF

%----------------------------------------------------------------------------------------
%	TITLE SECTION
%----------------------------------------------------------------------------------------

\setlength{\droptitle}{-4\baselineskip} % Move the title up

\pretitle{\begin{center}\Huge\bfseries} % Article title formatting
\posttitle{\end{center}} % Article title closing formatting
\title{EVOLUCIÓN TÉCNICA DE LAS REDES FISIOLÓGICAS} % Article title
\author{%
\textsc{Luis Esteban Ruelas} \\[1ex] % Your name
\normalsize Universidad Nacional Autónoma de México
 (Programa de Doctorado en Ciencias Biomédicas)\\ % Your institution
%\and % Uncomment if 2 authors are required, duplicate these 4 lines if more
%\textsc{Jane Smith}\thanks{Corresponding author} \\[1ex] % Second author's name
%\normalsize University of Utah \\ % Second author's institution
%\normalsize \href{mailto:jane@smith.com}{jane@smith.com} % Second author's email address
}
\date{28 de mayo de 2021} % Leave empty to omit a date
\renewcommand{\maketitlehookd}{%
\begin{abstract}
\noindent El campo de las redes fisiológicas es una rama de las ciencias biológicas de reciente creación, iniciado por el Dr. Plamen Ch. Ivanov en el año 2012 con su publicación "Network physiology reveals relations between network topology and physiological function" en la revista Nature Communications. Después de esta publicación, múltiples grupos de investigación han utilizado la ciencia de redes en sus respectivas áreas biológicas para obtener nuevas perspectivas multisistémicas e integrales en diversos estados fisiológicos. Las técnicas para obtener e interpretar estas redes han evolucionado a través del tiempo. Nos encontramos ante un campo prometedor en rápida y continua evolución.
\end{abstract}
}

%----------------------------------------------------------------------------------------

\begin{document}
\renewcommand{\abstractname}{Resumen}
\renewcommand{\figurename}{Figura}
% Print the title
\maketitle

%----------------------------------------------------------------------------------------
%	ARTICLE CONTENTS
%----------------------------------------------------------------------------------------

\section{Introducción}
En las ciencias médicas existen multitud de problemas de gran importancia, ya sea por la gravedad de los mismos, la gran cantidad de gente a la que afectan o ambas.
Tomemos por ejemplo una de las enfermedades cardiovasculares más importantes en México: la hipertensión arterial.
Causante de un 18.1\% de las muertes en México\cite{campos2018hipertension}, se diagnostica por medio de las cifras de presión arterial ()$\geq140/90$ al mes de una primera determinación de presión arterial\cite{de2008diagnostico});
Una vez instaurado el tratamiento (que se da basado en las condiciones del paciente como edad y comorbilidades) el control y éxito del tratamiento se evalúa, una vez más, con la presión arterial.
Si el paciente además tiene diabetes, se le realizan los estudios pertinentes a esta enfermedad. Si estuviéramos ante una enfermedad del riñón, se estudiaría con parámetros renales. En todo momento, la visión de cada uno de esos problemas se realiza desde la especialidad a la que corresponde, y realmente no existen enfoques objetivos que integren todos los sistemas implicados.

El cuerpo humano, como sistema complejo\cite{mobus2015principles}, es un conjunto inmenso de componentes en constante interacción \cite{engel2010thermodynamics}.
Como una propiedad de este tipo de sistemas, existe un llamado "punto crítico", donde coexiste el estado anterior (en el caso que nos atañe, un estado de no enfermedad) y otro estado diferente (enfermedad).
Es justo en este punto donde ocurre una transición de fase, donde el cuerpo pasaría a un estado patológico.
El enfoque reduccionista descrito en el párrafo anterior podrá diagnosticar la enfermedad solamente una vez que el punto crítico haya sido alcanzado y rebasado, por tanto, el sistema se encontrará en estado patológico.

Ésta es una de las necesidades que el campo de las redes fisiológicas puede tratar con naturalidad.
Ofrece forma de estudiar las interacciones entre diferentes sistemas que conforman el cuerpo humano, de forma que podamos distinguir diferentes estados fisiológicos.
El campo fue definido por el grupo del Dr. Plamen Ivanov en el año 2012 con un célebre artículo acerca de las diferencias entre las redes de 4 diferentes estados fisiológicos en el sueño y el estado de despierto\cite{bashan2012network}.
Desde entonces, las redes fisiológicas (construídas con diversas herramientas y técnicas), han ofrecido un enfoque integrativo en un número de enfermedades en constante incremento, utilizando sus métodos para separar estados fisiológicos por medio de todo tipo de parámetros como edad, sexo, variables de laboratorio entre otros\cite{ivanov2016focus}.

Aplicando las técnicas de teoría de redes se obtienen perspectivas nuevas acerca de los sistemas biológicos.
Ejemplo de estas técnicas son algoritmos de agrupamiento, que al reunir en grupos de nodos (llamados también comunidades) que tienen características comunes, nos dan información acerca de cómo funciona un sistema de forma interna, algo que, en muchos casos, está profundamente estudiado desde una perspectiva teórica.
Sin embargo, las conexiones entre diferentes comunidades nos ofrece un paradigma multisistémico, que otorga una perspectiva novedosa al abordar un sistema como parte un complejo más grande con el cual se relaciona de forma objetiva y del cual, generalmente, se aísla para su estudio en una disciplina en cuestión.
Para mencionar un caso, se pueden describir redes fisiológicas con enlaces entre señales electroencefalográficas y frecuencia cardiaca en diferentes estados fisiológicos, mostrando relación entre dos sistemas corporales (neurológico y cardiovascular) que generalmente se estudian de forma aislada\cite{campos2018hipertension}.

Siendo entonces las redes fisiológicas un campo novedoso, que promete soluciones que la visión reduccionista de los problemas no puede alcanzar con facilidad, surgen dos preguntas centrales:
\begin{itemize}
  \item ¿Qué métodos se utilizan para construir y analizar estas redes fisiológicas?
  \item ¿Qué problemas, dentro de la aplicabilidad biomédica, se han abordado utilizando ésta técnica?
\end{itemize}

La búsqueda de la respuesta a estas preguntas son el objeto de la siguiente revisión.

\subsection{Conceptos fundamentales para el estudio de redes fisiológicas}
Existen dos tipos de bases de datos de las cuales se obtendrán redes fisiológicas:
\begin{itemize}
  \item \textbf{Series de tiempo:} Estas incluyen una multitud de datos puntuales que representan el cambio de una variable fisiológica a través del tiempo.
  En el contexto de las redes fisiológicas, por ejemplo, un objetivo podría ser encontrar la relación entre una serie de tiempo (electroencefalograma) y otra serie de tiempo (electrocardiograma).
  En este tipo de base de datos, buscamos detalles acerca de la sincronización (retrasos de una señal con respecto a otra, correlación simple en el tiempo...).
  Cabe señalar que en este caso, ambas señales deberán provenir del mismo individuo, debido a que si no existe un mecanismo que las una, no podemos esperar que exista relación\cite{bashan2012network}.
  \item \textbf{Datos puntuales:} Un estado fisiológico puede estar representado por una serie de datos puntuales, por ejemplo la frecuencia cardiaca, la frecuencia respiratoria, la temperatura y los niveles de glucosa en un momento dado.
  Con este tipo de datos, no buscamos obtener detalles de sincronización entre las señales, sino observar si los datos se correlacionan en una población\cite{barajas2021physiological}.
\end{itemize}
Además, la creación e interpretación de redes fisiológicas es un proceso que implica la fusión de varios elementos \cite{barajas2021sex}, que incluyen:
\begin{itemize}
  \item \textbf{Identificación y definición de los nodos que conformarán la red:} Se definen las señales que formarán parte de la red a forma de nodos.
  \item \textbf{Métodos para definir los enlaces:} El método que se va a utilizar para acoplar las señales y así determinar los enlaces de la red.
  \item \textbf{Cálculo e interpretación de propiedades topológicas y dinámicas:} Una vez construida la red, se deben realizar los análisis dinámicos y topológicos pertinentes e interpretar los resultados.
\end{itemize}

\section{Métodos}
Debido al corto tiempo de vida del campo en cuestión, se decide revisar todos los artículos del Dr. Plamen Ivanov que incluyan redes fisiológicas a partir del año 2012 (inclusivo).
Para ser incluídos en la revisión, los artículos deben contener un método de construcción o análisis de redes fisiológicas inédito, o ser pertinentes a una nueva patología o estado fisiológico que no hubiese sido estudiado con anterioridad.
Se utiliza una búsqueda por autor (el Dr Plamen Ivanov) con el fin de no omitir artículos relevantes del fundador y principal publicador en el tema. Posteriormente se plantean en la búsqueda todos los campos que contengan la palabra clave "network". Al final la primera búsqueda queda planteada de la siguiente manera:

\texttt(Ivanov, Plamen Ch[Author]) AND (network) ,

Obteniendo como resultado 25 coincidencias. Se realiza una lectura diagonal de cada uno de los 25 artículos para obtener una perspectiva de los diferentes elementos históricos y técnicos que conforman redes fisiológicas.
Se descartan 2 artículos que no incluyen descripción de la métodología y 11 que no tocan el tema de redes fisiológicas. Por lo tanto quean 12 artículos que incluyen métodos para la construcción y análisis de redes fisiológicas. (Tabla \ref{tab:plamenGroup}).

Se plantea una segunda búsqueda con el término "'network physiology'", que da como resultado 243 resultados.

Se lee a través del título y resumen de todos ellos para  encontrar los que son pertinentes a la fisiología de redes en seres humanos y que no hubieran sido incluídos en la búsqueda anterior.
Se descartan 2 artículos que no incluyen descripción de la métodología, 18 por haber sido incluídos anteriormente y 211 que no tienen tocan la temática de redes fisiológicas.
Al final, 12 de los artículos cumplen con los criterios de inclusión (descritos en la Tabla \ref{tab:allGroups}).

Además se presta especial atención a las aplicaciones de fisiología de redes por parte de la UNAM en el centro de estudios de la complejidad C3, que ha sido impulsado por el Dr. A Barajas y la Dra. A.L. Rivera.

Una vez obtenido este conocimiento, se comienza la redacción de los resultados, dividiendo la temática en 4 subtítulos relevantes:
\begin{itemize}
  \item \textbf{Métodos para definir acoplamiento}: por la importancia que tiene en el campo, como iniciador y amplio uso dentro del campo de las redes fisiológicas.
  \item \textbf{Aplicación de teoría de redes}: para explicar las múltiples técnicas utilizados por los diversos grupos de investigación que han dedicado trabajos a las redes fisiológicas
  \item \textbf{Problemas abordados por medio de redes fisiológicas}: sección prevista para comentar la cantidad de aplicaciones dentro de los diversos campos biológicos en ciencias de la salud que pueden beneficiarse del estudio de las redes fisiológicas.
\end{itemize}
%------------------------------------------------

\section{Resultados}

\subsection{Series de tiempo y bases de datos puntuales}
De los 24 artículos revisados, hay 9 que manejan series de tiempo, 8 de ellos provienen del grupo del Dr. Plamen \cite{bashan2012network}\cite{bartsch2014coexisting}\cite{liu2015major}\cite{bartsch2015network}\cite{2015Plasticity}\cite{lin2016delay}\cite{rizzo2020network}\cite{lin2020dynamic},\cite{ivanov2021signal}
y 1 de un grupo diferente\cite{jansen2019network}).
Los 14 restantes resto (obviando las 2 revisiones bibliogŕaficas) trata de redes formadas a partir de bases de datos puntuales, incluyen solo un artículo del grupo del Dr. Plamen\cite{nakazato2020estimation} y 13 de otros grupos \cite{scala2014complex}\cite{chmiel2014spreading}\cite{monti2018network}\cite{pereira2018computational}\cite{pereira2019complex}\cite{zanetti2019information}\cite{barajas2020metabolic}\cite{lehnertz2020human}\cite{antonacci2020information}\cite{tan2020organ}\cite{barajas2021physiological}\cite{barajas2021sex}\cite{cohen2021robust}.

\subsection{Métodos para definir acoplamiento}
Los métodos utilizados para definir acoplamiento en series de tiempo incluyen TDS (8 usos por parte del Dr. Plamen\cite{bashan2012network}\cite{bartsch2014coexisting}\cite{liu2015major}\cite{bartsch2015network}\cite{2015Plasticity}\cite{lin2016delay}\cite{rizzo2020network}\cite{lin2020dynamic},\cite{ivanov2021signal}, 1 uso por parte de otros grupos \cite{jansen2019network}(\texttt{ver apéndice donde se describe el método para su cálculo}) y
el método de sincronizacioń de fase (1 uso por parte del grupo del Dr Plamen\cite{bartsch2014coexisting}, \texttt{(Ver apéndice donde se describe el método para su cálculo})

El resto de las investigaciones que no son series incluyen análisis de redes que se obtienen por medio de datos puntuales.
Las técnicas utilizadas para realizar acoplamiento en estas redes son las siguientes:
5 usos de correlación de Pearson\cite{nakazato2020estimation}\cite{scala2014complex}\cite{chmiel2014spreading}\cite{pereira2018computational}\cite{tan2020organ},
5 instancias de correlación de Spearman\cite{barajas2020metabolic}\cite{barajas2021physiological}\cite{barajas2021sex}\cite{pereira2018computational}\cite{nakazato2020estimation}.
2 de ANOVA  \cite{zanetti2019information}\cite{rizzo2020network},
y 1 instancia de Prueba F de Fischer \cite{zanetti2019information}.

\subsection{Aplicación de teoría de redes}
Existen diversas métricas para describir las propiedades de una red y para medir las diferencias entre dos o más redes.
Las medidas que muestran las propiedades de una red utilizadas en los trabajos que atañen a esta revisión son
propiedades topológicas como densidad \cite{barajas2021sex}\cite{barajas2021physiological}, reciprocidad \cite{barajas2021sex}\cite{barajas2021physiological}, coeficientes de agrupamiento\cite{monti2018network}\cite{barajas2021sex}\cite{barajas2021physiological}, vulnerabilidad a ataques dirigidos\cite{barajas2021physiological}, nodos por comunidad \cite{chmiel2014spreading}, total de enlaces \cite{tan2020organ}\cite{bashan2012network}, centralidad de intermediación \cite{tan2020organ}, fuerza de los enlaces\cite{pereira2018computational}\cite{bartsch2014coexisting}\cite{bartsch2015network} y distribución de grado\cite{monti2018network}\cite{lin2020dynamic}.

También se disciernen las técnicas de detección de comunidades como el algoritmo de Louvain\cite{barajas2021sex}\cite{barajas2021physiological}, MAP\cite{barajas2020metabolic}, Girvan-Newman\cite{scala2014complex}.

%------------------------------------------------
\subsection{Problemas abordados por medio de redes fisiológicas}
El abordaje de diferentes temas se ha realizado por medio de redes fisiológicas. Por supuesto, los estudios de sueño, que disciernen estados fisiológicos del sueño y el estado de despierto, son númerosos debido al grupo del Dr. Plamen (con un total de 10 estudios).
Otros campos incluyen: Odontología (maloclusión\cite{scala2014complex}, enfermedades renales\cite{tan2020organ}, insomnio\cite{jansen2019network}, Parkinson\cite{monti2018network}, estrés(inducido)\cite{zanetti2019information}, fisiología del deporte: 3 artículos\cite{balague2020network}\cite{pereira2018computational}\cite{pereira2019complex} y estudios de stados fisiológicos, incluidos estados fisiológicos por edad\cite{chmiel2014spreading}\cite{barajas2020metabolic}\cite{lehnertz2020human}\cite{barajas2021physiological} o sexo\cite{barajas2021sex}.
\section{Discusión}
En los primeros estudios del grupo de Dr. Plamen acerca de redes fisiológicas en humanos, existe una clara tendencia a utilizar series de tiempo, esto probablemente debido a que posee una excelente cohorte de pacientes sanos con estudio completo de polisomnografía\cite{bashan2012network}.
Sin embargo el resto de los grupos de investigación continúa en una dirección diferente, ya que las bases de datos que se utilizan en el mundo clínico, contienen mayormente datos puntuales (biomarcadores, signos vitales, estudios de imagen...), es entendible que métodos para crear redes fisiológicas con este tipo de base de datos hayan sido desarrollados.
Obtener bases de datos que incluyan series de tiempo para determinar estados fisiológicos debe ser una prioridad para los grupos de investigación que realicen investigaciones nuevas en el campo de fisiología de redes.
Por otro lado, si se continúan realizando investigaciones con datos previamente obtenidos (bases de datos nacionales, de seguridad social, expedientes clínicos...) lo más probable es que la tendencia continúe hacia la utilización de medidas puntuales.
Es probablemente por esta razón que vemos un decremento considerable en el uso de TDS por otros grupos, a pesar de ser uno de los parametros más confiables para determinar acoplamiento entre series de tiempo\cite{bartsch2014coexisting}.

En cuanto a las patologías abordadas, podemos ver que la fisiología del ejercicio ya cuenta con algunos artículos\cite{pereira2018computational}\cite{pereira2019complex} además de la revisión bibliográfica que plantea el campo\cite{balague2020network}.
En este campo no tenemos aún datos con series de tiempo, además no tenemos cohortes específicamente diseñadas con el propósito de estudiar el campo, se trata de un campo con gran potencial que aún no ha sido explotado.

En éste mismo rubro también encontramos enfermedades neurológicas como Parkinson\cite{monti2018network}, insomnio \cite{jansen2019network} y un esbozo por entender el cerebro en estrés\cite{zanetti2019information}.
Desde sus inicios el grupo de el Dr. Plamen se ha concentrado en una cohorte que discierne estados fisiológicos de sueño y el estado de despierto, por lo que ver abordajes que incluyan enfermedades neurológicas, y estados fisiológicos más allá de este contexto es interesante.
Un campo virgen y un siguiente paso lógico puede ser el estudio de enfermedades mentales más prevalentes como depresión y distimia por medio de redes fisiológicas. Estas podrían incluso realizarse con escalas validadas eque incluyan información puntual, como prueba de concepto (Hamilon, Goldberg...).

El campo de la odontología también se ha aprovechado de estas técnicas para crear modelos que muestran cómo (y potencialmente, \texttt{¿por qué?}) mejoran los pacientes con maloclusión\cite{scala2014complex}.

El esfuerzo por crear redes fisiológicas\cite{barajas2021physiological}\cite{barajas2021sex} que representen estados en personas sanas también es un hecho a considerar.
Es relevante considerar que en el futuro estas redes pueden usarse para comparar resultados encontrados en patologías, creando así un "Atlas de redes fisiológicas humano" (similar a la idea que plantea el grupo del Dr. Plamen en su revisión\cite{ivanov2016focus}).

Además, cada vez se hace un mayor énfasis en el análisis de redes.
Por medio de un mayor uso de métricas de teoría de redes que nos proveen información interesante acerca de los sistemas de estudios conforme avanza el campo de redes fisiológicas.
Gracias al desarrollo de metodologías que nos permiten utilizar bases de datos más pequeñas para crear redes fisiológicas, el alcance del campo también ha aumentado para los investigadores en el futuro \cite{cohen2021robust}.
Nos dirigimos hacia el objetivo de una red que integre la totalidad de funciones humanas\cite{barajas2021physiological}.

El campo de las redes fisiológicas provee una visión integral multisistémica.
Es interesante que aún no se hayan abordado las enfermedades que causan mayor mortalidad en el mundo por medio de este enfoque.
Enfermedades con alta prevalencia, morbimortalidad y esencialmente multiorgánicas (hipertensión arterial, diabetes, obesidad...), podrían ser abordadas por este campo con el objetivo de obtener perspectivas novedosas acerca del diagnóstico, tratamiento y pronóstico de estas patologías.
%----------------------------------------------------------------------------------------
%	REFERENCE LIST
%----------------------------------------------------------------------------------------

\onecolumn
\begin{center}
  \begin{longtable}{|p{.2\textwidth}p{.4\textwidth}p{.4\textwidth}|}
    \caption{Los 12 artículos publicados por el equipo de Plamen Ivanov concernientes al campo de redes fisiológicas, obtenidos tras la búsqueda "(Ivanov, Plamen Ch[Author]) AND (network) ".
    \texttt{TDS}: Time Delay Stability
    \texttt{CRPS}: Sincronización de fase cardiorrespiratoria
    \texttt{RSA}: Arritmia respiratoria
    \texttt{MCA}: Análisis de componente más grande
    \texttt{CCF}: Correlación cruzada
    \texttt{ANOVA}: análisis de varianza de un factor
    \texttt{PCA}: Análisis de componente principal
    }\\
    \hline
    \textbf{Referencia} & \textbf{Métodos} & \textbf{Población} \\
    \hline
    Bashan et al. (2012) \cite{bashan2012network} &
    TDS.
    Visualización de redes.
    Análisis de redes (topología: número de enlaces, fuerza de los enlaces, grado de los nodos, grado de los enlaces)
    &
    36 pacientes sanos (18 hombres , 18 mujeres, de 20 a 40 años de edad) sometidos a polisomnografía durante el sueño.
    \\ \hline
    Bartsch et al. (2014)\cite{bartsch2014coexisting} &
    TDS, CRPS, análisis de CRPS: Amplitud de la RSA, \% de sincronización, duración de periodos de sincronización.
    Visualización de redes.
    Análisis de redes: Fuerza de los enlaces.
    &
    36 pacientes sanos (18 hombres , 18 mujeres, de 20 a 40 años de edad) sometidos a polisomnografía durante el sueño.
    \\ \hline
    Liu et al. (2015)\cite{liu2015major} &
    TDS, MCA.
    Visualización de redes
    &
    36 pacientes sanos (18 hombres , 18 mujeres, de 20 a 40 años de edad) sometidos a polisomnografía durante el sueño.
    \\ \hline
    Bartsch et al. (2015)\cite{bartsch2015network} &
    TDS.
    Visualización de redes.
    Análisis de redes: Fuerza de los enlaces.
    &
    36 pacientes sanos (18 hombres , 18 mujeres, de 20 a 40 años de edad) sometidos a polisomnografía durante el sueño.
    \\ \hline
    Lin et al. (2015) \cite{2015Plasticity} &
    TDS.
    Visualización de redes.
    &
    36 pacientes sanos (18 hombres , 18 mujeres, de 20 a 40 años de edad) sometidos a polisomnografía durante el sueño.
    \\ \hline
    Lin et al. (2016)\cite{lin2016delay} &
    CC,
    Mapa de retraso-correlación, TDS
    &
    34 pacientes sanos (17 hombres, 17 mujeres, entre 20 y 40 años). Medidos con polisomnografía durante el sueño nocturno.
    \\ \hline
    Lin et al. (2016)\cite{ivanov2016focus} &
    N/A
    &
    N/A
    \\ \hline
    Rizzo et al. (2020)\cite{rizzo2020network} &
    TDS.
    Análisis estadístico: ANOVA (aplicado a los rangos).
    Análisis de redes: fuerza promedio de los enlaces
    &
    36 pacientes sanos (18 hombres , 18 mujeres, de 20 a 40 años de edad) sometidos a polisomnografía durante el sueño.
    \\ \hline
    Lin et al. (2020)\cite{lin2020dynamic} &
    TDS, CC, Distribución de valores de CC, decomposición espectral de ondas.
    &
    34 pacientes sanos (17 hombres, 17 mujeres, entre 20 y 40 años). Medidos con polisomnografía durante el sueño nocturno.
    \\ \hline
    Balagué et al. (2020)\cite{balague2020network} &
    N/A
    &
    N/A
    \\ \hline
    Nakazato et al. (2020)\cite{nakazato2020estimation} &
    Correlación de Spearman, correlación de Pearson, prueba exacta de Fischer.
    PCA.
    &
    580 pacientes con más de 6 meses en tratamiento de hemodiálisis. 140 fueron reclutados Análisis de fragilidad.
    \\ \hline
    Ivanov et al. (2021)\cite{ivanov2021signal} &
    TDS.
    Análisis de redes: fuerza de los enlaces por \% de TDS.
    &
    52 pacientes sanos, 26 hombres, 26 mujeres, de entre 20 y 34 años. Medicos con polisomnografía durante el sueño nocturno.
    \\ \hline
    \label{tab:plamenGroup}
  \end{longtable}
\end{center}
\begin{center}
  \begin{longtable}{|p{0.2\textwidth}p{0.4\textwidth}p{0.4\textwidth}|}
    \caption{Tabla que incluye los 12 artíclos tras la búsqueda "network physiology". También incluye los del grupo de investigación C3 de la Universidad Nacional Autónoma de México}\\
    \hline
    \textbf{Estudio} & \textbf{Objetivo} & \textbf{Novedad en el campo} \\
    \hline
    Scala et al. (2014) \cite{scala2014complex}&
    Medición de 21 puntos anatómicos en resonancia magnética nuclear.
    Correlación de Pearson.
    Girvan–Newman para detección de comunidades.
    &
    70 pacientes con protrusión mandibular (maloclusión grado III) 40 hombres, 30 mujeres, de entre 7 y 13 años.
    \\\hline
    Chmiel et al. (2014) \cite{chmiel2014spreading} &
    Medidas estadísticas: prevalencia por grupo de edad, riesgo condicionado.
    Correlación de Pearson.
    Conteo de nodos por comunidad.
    &
    Base de datos de la Asociación principal de Seguridad Social de Austria. 1,862,258 pacientes, 1,064,952 mujeres, 797,306 hombres.
    \\\hline
    Monti et al. (2018) \cite{monti2018network} &
    Análisis de rutas metabólicas, minería de cadenas de texto.
    Anlisis de redes: distribución de grado, coeficientes de agrupamientos, caminos más cortos.
    &
    Busqueda de terminos proteom*[Title/Abstract]) AND Parkinson* [Title/Abstract] en PubMed.
    \\\hline
    Pereira et al. (2018) \cite{pereira2018computational} &
    Correlación de Pearson, correlación de Spearman.
    Análisis de redes: centralidad de intermediación. Grado específico.
    &
    8 corredores profesionales de entre entre 18 y 24 años de edad con lesiones debidas que involucren tratamiento médico.
    \\\hline
    Jansen et al. (2019) \cite{jansen2019network} &
    TDS, \% de TDS.
    Análisis de redes: Redes neuronales, árboles de decisión. Conexión específica.
    &
    64 pacientes con insomnio (41 mujeres, 23 hombres), media 51 años. Datos de polisomnografía.
    197 sujetos sanos. Datos de polisomnografía.
    \\\hline
    Pereira et al. (2019) \cite{pereira2019complex} &
    Correlación de Pearson.
    Calculo de eigenvalores.
    Análisis de la red: Fuerza de enlaces
    &
    40 atletas de entre 23 y 29 años, hombres, nadadores de estilo libre de 50m.
    \\\hline
    Zanetti et al. (2019) \cite{zanetti2019information} &
    Análisis estadístico: Prueba F de Fischer. ANOVA.
    Análisis de la red: fuerza de enlaces, conexión específica.
    &
    18 participantes entre 18 y 30 años. 3 grupos: relajación, estrés inducido por aritmética y juego serio.
    \\\hline
    Antonacci et al. (2020) \cite{antonacci2020information} &
    Modelo autoregresivo Identificador de vectores.
    Medidas de transferencia de información.
    Medidas de entropía.
    &
    Simulación de diferentes escenarios.
    \\\hline
    Tan et al. (2020) \cite{tan2020organ} &
    Correlación de Pearson.
    Análisis de la red: Total de enlaces. Conectividad promedio. Centralidad de intermediación.
    &
    201 pacientes, 156 hombres, 145 mujeres con cirrosis hepática sin carcinoma hepatocelular. Divididos en supervivientes y no supervivientes.
    \\ \hline
    Barajas et al. (2020) \cite{barajas2020metabolic} &
    Correlación de Spearman.
    Eigencentralidad
    Análisis de red: Análisis de componentes (Louvain y MAP), medidas de topología: densidad, reciprocidad, tamaño de los caminos, transitividad, coeficiente de agrupamiento.
    &
    2572 entre 18 y 81 años de la ciudad de México, dividida en grupos de edad, con un promedio de 65\% mujeres.
    \\\hline
    Barajas-Martinez et al. (2021) \cite{barajas2021physiological}
    &
    Correlación de Spearman.
    Eigencentralidad
    Análisis de red: Análisis de componentes (Louvain), medidas de topología: densidad, reciprocidad, tamaño de los caminos, transitividad, coeficiente de agrupamiento.
    &
    Análisis cardiaco: 10 hombres y 10 mujeres entre 21 y 34 años de edad; 10 hombres y 10 mujeres de entre 65 y 85 años de edad, medidos con 120mins de electrocardiograma.
    Análisis bioquímico: dos bases de datos, la primera del Instituto Nacional de Enfermedades Respiratorias (INER), 134 pacientes, 43 hombres y 91 mujeres, de entre 25 y 67 años; Estudiantes de primer y segundo año de la facultad de medicina UNAM, 844 participantes 281 hombres y 563 mujeres
    \\\hline
    Barajas-Martinez et al. (2021) \cite{barajas2021sex}
    &
    Correlación de grado de Spearman, análisis de eigenvectores.
    Análisis de redes: detección de comunidades (Louvain), número de enlaces, fuerza promedio de enlaces, centralidad de nodos, centralidad de flujo, densidad, reciprocidad, transitividad,coeficientes de agrupamiento (local y global).
    Vulnerabilidad a ataques dirigidos.
    &
    Estudiantes de primer y segundo año de la facultad de medicina UNAM, 844 participantes 281 hombres y 563 mujeres
    30 redes creadas para mujeres y 30 para hombres, de entre 117 individuos seleccionados al azar.
    \\\hline
    \label{tab:allGroups}
  \end{longtable}
\end{center}
\section{Apéndices}
\subsection{Time Delay Stability}
El análisis del Time Delay Stability (TDS) es un recurso que puede aplicarse a las series de tiempo.
Se trata de la primera innovación propuesta por el doctor Plamen Ivanov en el año 2012 \cite{bashan2012network}, los sistemas fisiológicos funcionan a través de un conjunto de retroalimentaciones positivas y negativas, donde el aumento en una de las señales debería llevar al aumento o disminución de las señales \textit{acopladas} a ella. Es posible medir esto a través de correlación simpĺe, pero en ese caso se perdería la relación si los cambios en una señal tardaran un tiempo determinadas en verse reflejadas en la otra (aún cuando este tiempo de "desfase" fuera constante).
Para resolver este problema, se puede recurrir a la segmentación de las señales y el cálculo de su función de correlación dándonos así el desfase de una señal con respecto a otra. Pero la fuerza del método de la TDS está en tomar los valores $\tau$ de la función de correlación y compararlos en sí, creando entonces una evaluación de su \textit{estabilidad}.

Para el cálculo de la TDS, el primer paso para el cálculo de esta métrica consiste en tomar dos series de tiempo correspondientes a dos variables fisiológicas de interés $x$ y $y$ con un número de puntos $N$ que dividiremos en una número $N_L$ de segmentos $v$ parcialmente superpuestos, todos ellos de igual longitud $L$.
La superposición puede definirse como convenga, en este ejemplo usaremos $L/2$
Posteriormente se normalizará la serie de tiempo de cada segmento $v_{1...N_L}$ a media cero y desviación estándar unitaria por medio de la siguiente fórmula:
\begin{equation}
v^\prime_t = \frac{x_t-\bar{x}}{\sigma} ,
\end{equation}
esto con el fin de poder utilizar la siguiente función de correlación sobre cada uno de los segmentos, obtenido posterior a la normalización:
\begin{equation}
C^v_{xy}(\tau)=\frac{1}{L}\sum^L_{i=1}x^v_{i+(v-1)L/2}y^v_{i+(v-1)L/2+\tau} ,
\end{equation}
Se define así $\tau^v_0$ como el retraso de tiempo $\tau$ que maximice la función $C^v_{xy}$ para este segmento $v$, y definiremos una nueva serie de tiempo ${\tau^v_0}=1,...,N_L$.
Consideraremos que ambas series están acopladas en los periodos de tiempo donde los valores de la nueva serie $\tau_0$ se mantengan constantes.
El TDS proporciona una medida fidedigna de acoplamiento entre dos señales y es posible utilizarla entre dos series con diferentes unidades (por ejemplo: frecuencia cardiaca -latidos por minuto- y presión arterial -milímiteros de mercurio) e incluso diferente magnitudes (frecuencia respiratoria -en $1x10^1$ y señal electrocardiográfica - en $1x10^{-3}V$).

Se ha mostrado que los sistemas fisiológicos pueden tener dos o más acoplamientos simultáneos y que operan a diferentes escalas.
Por lo tanto se ha propuesto utilizar diferentes métodos de acoplamiento en las señales, para capturar estos acoplamientos que pueden resultar invisibles para alguno de los métodos.
\subsection{Sincronización de fase}
Si dos sistemas fisiológicos tienen dos métodos de acoplamiento que funcionan a diferentes escalas temporales (con órdenes de magnitud de diferencia) será imposible que un sólo método (TDS) pueda detectarlos.
Un ejemplo de esto es la frecuencia cardiaca, y su acoplamiento con la frecuencia respiratoria \cite{bartsch2014coexisting} sobre las cuales se puede definir un acoplamiento basados en el método de \texttt{sincronización de fase} (PS) que describiremos a continuación.
Las señales utilizadas pueden ser el flujo de aire nasal o el perímetro torácico, incluso puede crearse una nueva señal mezclando los mejores puntos de ambas \cite{bartsch2014coexisting}, que se utilizarán para crear una onda sinusal que represente el ciclo de frecuencia respiratoria como la Figura \ref{fig:sinusoidalBreathing}.
\begin{figure}[H]
\includegraphics[width=8cm]{sinusoidalBreathing.png}
\caption{Se puede observar claramente la correspondencia 3:1 entre los oscilantes ciclos respiratorios -arriba- y la frecuencia cardiaca -abajo. }
\label{fig:sinusoidalBreathing}
\end{figure}
Uno de los procesos para evaluar acoplamiento en señales no lineales es evaluarlas a través de sincrogramas \cite{bartsch2014coexisting}.
En este caso se van a graficar los eventos de una segunda señal no oscilatoria (en este caso electrocardiograma) sobre una señal continua oscilatoria que es la señal de frecuencia respiratoria $r(t)$. El evento de segunda señal se puede definir como el punto más alto del complejo QRS (pico de onda R) en el electrocardiograma $t_k$.
Una vez hecho esto se puede identificar la fase respiratoria instantanea $\phi_r(t)$, correspondiente a un evento de electrocardiograma determinado. Así  $\phi_r(t)$ representará el ángulo de la señal $r(t)$ y su transformada de Hilbert $r_H(t)$ que a su vez corresponde a la parte imaginaria de $\phi_r(t)$.
La gráfica de ángulos $\phi_r(t_k)$ nos otorga el sincrograma que se puede observar en la Figura \ref{fig:synchrogram}. Se define entonces el acoplamiento si existe un número $n$ de líneas paralelas, donde $n$ es el número de latidos por ciclo respiratorio, como en la Figura \ref{fig:linePlotting}.

\begin{figure}[H]
\includegraphics[width=8cm]{linePlotting.png}
\caption{Grafica de 3 líneas horizontales paralelas formadas, formadas por el primer, segundo y tercer latido dentro de 3 ciclos respiratorios consecutivos. Se puede ver claramente como hay una sincronización 3:1, con 3 latidos cardiacos en cada ciclo respiratorio}
\label{fig:linePlotting}
\end{figure}

\begin{figure}[H]
\includegraphics[width=7cm]{synchrogram.png}
\caption{Sincrograma, sobre la señal respiratoria oscilante (3 ciclos en diferentes colores: rojo, azul y negro), se grafican puntos incidentes de electrocardiograma $t_k$ en los ángulos de fase $\phi_r(t_k)$\cite{bartsch2014coexisting}}
\label{fig:synchrogram}
\end{figure}
Así pues se obtienen un método de acoplamiento que puede ver sincronizaciones diferentes a las del TDS, como se puede observar en la Figura \ref{fig:coexistingFormOfCoupling} que evidencía acoplamiento en TDS (azul) diferente a la PS en (rojo).
\begin{figure}[H]
\includegraphics[width=8cm]{coexistingFormOfCoupling.png}
\caption{Se pueden observar diferentes métodos de coupling existentes entre las mismas señales. En rojo PS, en azul TDS.}
\label{fig:coexistingFormOfCoupling}
\end{figure}
\renewcommand\refname{Referencias}
\printbibliography
\end{document}
