%%%%%%%%%%%%%%%%%%%%%%%%%%%%%%%%%%%%%%%%%
% Journal Article
% LaTeX Template
% Version 1.4 (15/5/16)
%
% This template has been downloaded from:
% http://www.LaTeXTemplates.com
%
% Original author:
% Frits Wenneker (http://www.howtotex.com) with extensive modifications by
% Vel (vel@LaTeXTemplates.com)
%
% License:
% CC BY-NC-SA 3.0 (http://creativecommons.org/licenses/by-nc-sa/3.0/)
%
%%%%%%%%%%%%%%%%%%%%%%%%%%%%%%%%%%%%%%%%%

%----------------------------------------------------------------------------------------
%	PACKAGES AND OTHER DOCUMENT CONFIGURATIONS
%----------------------------------------------------------------------------------------

\documentclass[twoside,twocolumn]{article}
\usepackage[sorting=none]{biblatex}
\bibliography{Bibliography.bib}
\usepackage{graphicx}
\usepackage{csquotes}
\usepackage{blindtext} % Package to generate dummy text throughout this template

\usepackage[sc]{mathpazo} % Use the Palatino font
\usepackage[T1]{fontenc} % Use 8-bit encoding that has 256 glyphs
\linespread{1.05} % Line spacing - Palatino needs more space between lines
\usepackage{microtype} % Slightly tweak font spacing for aesthetics

\usepackage[english]{babel} % Language hyphenation and typographical rules

\usepackage[hmarginratio=1:1,top=32mm,columnsep=20pt]{geometry} % Document margins
\usepackage[hang, small,labelfont=bf,up,textfont=it,up]{caption} % Custom captions under/above floats in tables or figures
\usepackage{booktabs} % Horizontal rules in tables

\usepackage{lettrine} % The lettrine is the first enlarged letter at the beginning of the text

\usepackage{enumitem} % Customized lists
\setlist[itemize]{noitemsep} % Make itemize lists more compact

\usepackage{abstract} % Allows abstract customization
\renewcommand{\abstractnamefont}{\normalfont\bfseries} % Set the "Abstract" text to bold
\renewcommand{\abstracttextfont}{\normalfont\small\itshape} % Set the abstract itself to small italic text

\usepackage{titlesec} % Allows customization of titles
\renewcommand\thesection{\Roman{section}} % Roman numerals for the sections
\renewcommand\thesubsection{\roman{subsection}} % roman numerals for subsections
\titleformat{\section}[block]{\large\scshape\centering}{\thesection.}{1em}{} % Change the look of the section titles
\titleformat{\subsection}[block]{\large}{\thesubsection.}{1em}{} % Change the look of the section titles

\usepackage{fancyhdr} % Headers and footers
\pagestyle{fancy} % All pages have headers and footers
\fancyhead{} % Blank out the default header
\fancyfoot{} % Blank out the default footer
\fancyhead[C]{Running title $\bullet$ May 2016 $\bullet$ Vol. XXI, No. 1} % Custom header text
\fancyfoot[RO,LE]{\thepage} % Custom footer text

\usepackage{titling} % Customizing the title section

\usepackage{hyperref} % For hyperlinks in the PDF

%----------------------------------------------------------------------------------------
%	TITLE SECTION
%----------------------------------------------------------------------------------------

\setlength{\droptitle}{-4\baselineskip} % Move the title up

\pretitle{\begin{center}\Huge\bfseries} % Article title formatting
\posttitle{\end{center}} % Article title closing formatting
\title{ANTECEDENTES TÉNICOS E HISTÓRICOS DE LAS REDES FISIOLÓGICAS} % Article title
\author{%
\textsc{Ruben Yvan Marteen Fossion} \\[1ex]
\textsc{Claudia Lerma Gonzalez} \\[1ex]
\textsc{Jesus Espinal Enriquez} \\[1ex] % Your name
\textsc{Luis Esteban Ruelas} \\[1ex] % Your name
\normalsize Universidad Nacional Autónoma de México
 (Programa de Doctorado en Ciencias Biomédicas)\\ % Your institution
%\and % Uncomment if 2 authors are required, duplicate these 4 lines if more
%\textsc{Jane Smith}\thanks{Corresponding author} \\[1ex] % Second author's name
%\normalsize University of Utah \\ % Second author's institution
%\normalsize \href{mailto:jane@smith.com}{jane@smith.com} % Second author's email address
}
\date{28 de mayo de 2021} % Leave empty to omit a date
\renewcommand{\maketitlehookd}{%
\begin{abstract}
\noindent \blindtext % Dummy abstract text - replace \blindtext with your abstract text
\end{abstract}
}

%----------------------------------------------------------------------------------------

\begin{document}
\renewcommand{\abstractname}{Resumen}
% Print the title
\maketitle

%----------------------------------------------------------------------------------------
%	ARTICLE CONTENTS
%----------------------------------------------------------------------------------------

\section{Introducción}

%------------------------------------------------

\section{Métodos}

%------------------------------------------------

\section{Resultados}
La creación e interpretación de redes fisiológicas es un proceso que implica la fusión de varios elementos \cite{barajas2021sex}, que incluyen:
\begin{itemize}
  \item \textbf{Identificacíón y definición de los nodos que conformarán la red:} Se definen las señales que formarán parte de la red a forma de nodos.
  \item \textbf{Métodos para definir los enlaces:} El método que se va a utilizar para acoplar las señales y así determinar los enlaces de la red.
  \item \textbf{Cálculo e interpretación de propiedades topológicas y dinámicas:} Una vez construida la red, se deben realizar los análisis dinámicos y topográficos pertinentes e interpetar los resultados.
\end{itemize}
Para cumplir con estos tres objetivos se han utilizado diversas téncnicas a través de los años, las cuales se describen a continuación.
\subsection{Time Delay Stability (TDS)}
La primera innovación propuesta por el doctor Plamen Ivanov en el año 2012 es el TDS \cite{bashan2012network}, los sistemas fisiológicos funcionan a través de un conjunto de retroalimentaciones positivas y negativas, donde el aumento en una de las señales debería llevar al aumento o disminución de las señales \textit{acopladas} a ella. Es posible medir estto a través de correlación simpĺe, pero en ese caso se perdería la relación si estas señales están desfasadas en el tiempo.
Para resolver este problema, se puede recurrir a la segmentación de las señales y el calculo de su función de correlación dandonos así el desfase de una señal con respecto a otra. Pero la fuerza del método de la TDD está en tomar los valores $\tau$ de la función de correlación y compararlos en sí, creando entonces una evaluación de su \textit{estabilidad}.

Para el cálculo de la TDD, el primer paso para el cálculo de esta métrica consiste en tomar dos series de tiempo correspondientes a dos variables fisiológicas de interés $x$ y $y$ de longitud que dividiremos en una cantidad $N_L$ de segmentos $v$ parcialmente superpuestos, todos ellos de igual longitud $L$.
La superposición puede definirse como convenga, en este ejemplo usaremos $L/2$
Posteriormente se normalizará la serie de tiempo de cada segmento $v$ a media cero y unidad de desviación estandar por medio de la siguiente fórmula:
\begin{equation}
  v_t = \frac{v_t-\bar{v}}{\sigma} ,
\end{equation}
esto con el fin de poder utilizar la siguiente función de correlación sobre el segmento $v = 1,...,N_L$ obtenido posterior a la normalización:
\begin{equation}
  C^v_{xy}\tau=\frac{1}{L}\sum^L_{i=1}x^v_{i+(v-1)L/2}y^v_{i+(v-1)L/2+\tau} ,
\end{equation}
Definiremos así $\tau^v_0$ como el retraso de tiempo $\tau$ que maximice la función $C^v_{xy}$ para este segmento $v$, y definiremos una nueva serie de tiempo ${\tau^v_0}_v=1,...,N_L$.
Consideraremos que amabas series están acopladas en los periodos de tiempo donde los valores de la nueva serie $\tau0$ se mantengan contastes.
Aunque el TDS proporciona una medida fidedigna de acoplamiento entre dos señales y es posible utilizarla entre dos series con diferentes unidades e incluso diferente magnitudes, se ha mostrado que los sitemas fisiológicos pueden tener dos o más acoplamientes simultaneos y que operan a diferentes escalas.
Por lo tanto se ha propuesto utilizar diferentes métodos de acoplamiento en las señales, para capturar estos acoplamientos que pueden resultar invisibles para alguno de los métodos.

\subsection{Diferentes metodos de acoplamiento}
Si dos sistemas fisiológicos tienen dos métodos de acoplamiento que funcionan a diferentes escalas será imposible que un sólo metodo (TDS) pueda detecarlos.

Un ejemplo de esto es la frecuencia cardiaca, y su acoplamiento con la frecuencia respiratoria \cite{bartsch2014coexisting}.
Las señales utilizadas pueden ser el flujo de aire nasal o el perímetro torácico, incluso puede crearse una nueva señal mezclando los mejores puntos de ambas \cite{bartsch2014coexisting}, que se utilizarán para crear una onda sinusal que represente el ciclo de frecuencia respiratoria como la Figura \ref{fig:sinusoidalBreathing}.

Uno de los procesos para evaluar acoplamiento en señales no lineales es evaluarlas a traves de sincrogramas.
En este caso se van a graficar los eventos de la segunda señal no oscilatoria (en este caso electrocardiograma) sobre una señal continua oscilatoria que es la señal de frecuencia respiratoria $r(t)$. El evento de segunda señal se puede definir como los picos de R $t_k$.
Una vez hecho esto se puede identificar la fase respiratoria instantanea $\phi_r(t)$, correspondiente a un evento de electrocardiograma determinado. Así  $\phi_r(t)$ representará el angulo de la señal $r(t)$ y su transformada de Hilbert $r_H(t)$ que a su vez corresponde a la parte imaginaria de $\phi_r(t)$.
La gráfica de angulos $\phi_r(t_k)$ nos otorga el sincrograma que se puede observar en la Figura \ref{fig:synchrogram}. Se define entonces el acoplamiento si existe un numero $n$ de lineas paralelas, donde $n$ es el número de latidos por ciclo respiratorio, como en la Figura \ref{fig:linePlotting}.

Así pues se obtienen un método de acoplamiento que puede ver sincronizaciones diferentes a las del TDS, como se puede observar en la Figura \ref{coexistingFormOfCoupling.png} que evidencía acoplamiento en TDS (azul) diferente al PS (rojo).
%------------------------------------------------

\section{Discusion}

%----------------------------------------------------------------------------------------
%	REFERENCE LIST
%----------------------------------------------------------------------------------------
\renewcommand\refname{Referencias}
\printbibliography
\begin{figure}
  \includegraphics[width=8cm]{coexistingFormOfCoupling.png}
  \caption{Se pueden observar diferentes métodos de coupling existentes entre las mismas señales. En rojo PS, en azul TDS.}
  \label{fig:coexistingFormOfCoupling}
\end{figure}
\begin{figure}
  \includegraphics[width=8cm]{sinusoidalBreathing.png}
  \caption{Se pueden observar diferentes métodos de coupling existentes entre las mismas señales. En rojo PS, en azul TDS.}
  \label{fig:sinusoidalBreathing}
\end{figure}
\begin{figure}
  \includegraphics[width=8cm]{synchrogram.png}
  \caption{Sincrograma, sobre la señal oscilante (señal respiratoria), se grafican puntos incidentes de electrocardiograma $t_k$ en los ángulos de fase $phi_r(t_k)$}
  \label{fig:synchrogram}
\end{figure}
\begin{figure}
  \includegraphics[width=8cm]{linePlotting.png}
  \caption{Grafica de 3 líneas horizontales paralelas formadas, formadas por el primer, segundo y tercer latido dentro de 3 ciclos respiratorios consecutivos. Se puede ver claramente como hay una sincronización 3:1, con 3 latidos cardiacos en cada ciclo respiratorio}
  \label{fig:linePlotting}
\end{figure}
\end{document}
