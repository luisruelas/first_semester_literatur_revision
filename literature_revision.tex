%%%%%%%%%%%%%%%%%%%%%%%%%%%%%%%%%%%%%%%%%
% Journal Article
% LaTeX Template
% Version 1.4 (15/5/16)
%
% This template has been downloaded from:
% http://www.LaTeXTemplates.com
%
% Original author:
% Frits Wenneker (http://www.howtotex.com) with extensive modifications by
% Vel (vel@LaTeXTemplates.com)
%
% License:
% CC BY-NC-SA 3.0 (http://creativecommons.org/licenses/by-nc-sa/3.0/)
%
%%%%%%%%%%%%%%%%%%%%%%%%%%%%%%%%%%%%%%%%%

%----------------------------------------------------------------------------------------
%	PACKAGES AND OTHER DOCUMENT CONFIGURATIONS
%----------------------------------------------------------------------------------------

\documentclass[twoside,twocolumn]{article}
\usepackage[sorting=none]{biblatex}
\bibliography{Bibliography.bib}
\usepackage{graphicx}
\usepackage{float}
\usepackage{csquotes}
\usepackage{blindtext} % Package to generate dummy text throughout this template

\usepackage[sc]{mathpazo} % Use the Palatino font
\usepackage[T1]{fontenc} % Use 8-bit encoding that has 256 glyphs
\linespread{1.05} % Line spacing - Palatino needs more space between lines
\usepackage{microtype} % Slightly tweak font spacing for aesthetics

\usepackage[english]{babel} % Language hyphenation and typographical rules

\usepackage[hmarginratio=1:1,top=32mm,columnsep=20pt]{geometry} % Document margins
\usepackage[hang, small,labelfont=bf,up,textfont=it,up]{caption} % Custom captions under/above floats in tables or figures
\captionsetup[table]{name=Tabla}
\usepackage{booktabs} % Horizontal rules in tables

\usepackage{lettrine} % The lettrine is the first enlarged letter at the beginning of the text

\usepackage{enumitem} % Customized lists
\setlist[itemize]{noitemsep} % Make itemize lists more compact

\usepackage{abstract} % Allows abstract customization
\renewcommand{\abstractnamefont}{\normalfont\bfseries} % Set the "Abstract" text to bold
\renewcommand{\abstracttextfont}{\normalfont\small\itshape} % Set the abstract itself to small italic text

\usepackage{titlesec} % Allows customization of titles
\renewcommand\thesection{\Roman{section}} % Roman numerals for the sections
\renewcommand\thesubsection{\roman{subsection}} % roman numerals for subsections
\titleformat{\section}[block]{\large\scshape\centering}{\thesection.}{1em}{} % Change the look of the section titles
\titleformat{\subsection}[block]{\large}{\thesubsection.}{1em}{} % Change the look of the section titles

\usepackage{fancyhdr} % Headers and footers
\pagestyle{fancy} % All pages have headers and footers
\fancyhead{} % Blank out the default header
\fancyfoot{} % Blank out the default footer
%\fancyhead[C]{Running title $\bullet$ May 2016 $\bullet$ Vol. XXI, No. 1} % Custom header text
\fancyfoot[RO,LE]{\thepage} % Custom footer text

\usepackage{titling} % Customizing the title section

\usepackage{hyperref} % For hyperlinks in the PDF

%----------------------------------------------------------------------------------------
%	TITLE SECTION
%----------------------------------------------------------------------------------------

\setlength{\droptitle}{-4\baselineskip} % Move the title up

\pretitle{\begin{center}\Huge\bfseries} % Article title formatting
\posttitle{\end{center}} % Article title closing formatting
\title{EVOLUCIÓN TÉCNICA DE LAS REDES FISIOLÓGICAS} % Article title
\author{%
\textsc{Luis Esteban Ruelas} \\[1ex] % Your name
\normalsize Universidad Nacional Autónoma de México
 (Programa de Doctorado en Ciencias Biomédicas)\\ % Your institution
%\and % Uncomment if 2 authors are required, duplicate these 4 lines if more
%\textsc{Jane Smith}\thanks{Corresponding author} \\[1ex] % Second author's name
%\normalsize University of Utah \\ % Second author's institution
%\normalsize \href{mailto:jane@smith.com}{jane@smith.com} % Second author's email address
}
\date{28 de mayo de 2021} % Leave empty to omit a date
\renewcommand{\maketitlehookd}{%
\begin{abstract}
\noindent El campo de las redes fisiológicas es una rama de las ciencias biológicas de reciente creación, iniciado por el Dr. Plamen Ch. Ivanov en el año 2012 con su publicación "Network physiology reveals relations between network topology and physiological function" en la revista Nature Communications. Después de esta publicación, múltiples grupos de investigación han utilizado la ciencia de redes en sus respectivas áreas biológicas para obtener nuevas perspectivas multisistémicas e integrales en diversos estados fisiológicos. Las técnicas para obtener e interpretar éstas redes han evolucionado a través del tiempo. Nos encontramos ante un campo prometedor en rápida y continua evolución.
\end{abstract}
}

%----------------------------------------------------------------------------------------

\begin{document}
\renewcommand{\abstractname}{Resumen}
\renewcommand{\figurename}{Figura}
% Print the title
\maketitle

%----------------------------------------------------------------------------------------
%	ARTICLE CONTENTS
%----------------------------------------------------------------------------------------

\section{Introducción}
En las ciencias médicas existen una multitud de problemas de gran importancia, ya sea por la gravedad de los mismos, la gran cantidad de gente a la que afectan o ambas.
Tomemos por ejemplo una de las enfermedades cardiovasculares más importantes en México: la hipertensión arterial. Causante de un 18.1\% de las muertes en México\cite{campos2018hipertension}, se diagnostica por medio de las cifras de presión arterial, una vez instaurado el tratamiento (que se da basado en las condiciones del paciente como edad y comorbilidades) el control y éxito del tratamiento se evalúa, una vez más, con la presión arterial.
Si el paciente además tiene diabetes, se le realizan los estudios pertinentes a esta enfermedad. Si hubiera infermedad del riñón, se estudiaría con parámetros renales. En todo momento, la visión de cada uno de esos problemas se realiza desde la especialidad a la que corresponde, y realmente no existen enfoques objetivos que integren todos los sistemas implicados.

Es de ésta necesidad que nace el campo de las redes fisiológicas, definido por el grupo del Dr. Plamen Ivanov en el año 2012 con un célebre artículo acerca de las diferencias entre las redes de 4 diferentes estados fisiológicos en el sueño y el estado de despierto\cite{bartsch2014coexisting}.
Las redes fisiológicas, por medio de diversas técnicas, conectan las variables de múltiples sistemas fisiológicos y convierten esta información en una red que representa las relaciones entre las diferentes variables, dándonos así información de cómo opera el sistema de manera interna, pero a su vez, nos ofrece información acerca de su relación con el resto de los sistemas de la red.

Después de éste artículo se han abierto una multitud de aplicaciones para esta nueva rama de estudio en diversas áreas de las ciencias de la salud.
Muchos estudios han sido publicados utilizando la fisiología de redes para separar estados fisiológicos utilizando todo tipo de parámetros como edad, sexo, variables de laboratorio entre otros.

Aplicando las técnicas de teoría de redes se obtienen perspectivas nuevas acerca de los sistemas biológicos.
Ejemplo de estás técnicas son algoritmos de agrupamiento, que al reunir en grupos de nodos (llamados también comunidades) que tienen características comunes, nos dan información acerca de cómo funciona un sistema de forma interna, algo que, en muchos casos, está profundamente estudiado desde una perspectiva teórica.
Sin embargo las conexiones entre diferentes comunidades nos ofrece un paradigma multisistémico, que otorga una perspectiva novedosa al abordar un sistema como parte un complejo más grande con el cual se relaciona de forma objetiva y del cual, generalmente, se aísla para su estudio en una disciplina en cuestión.
Para mencionar un caso, se pueden describir redes fisiológicas con enlaces entre señales electroencefalográficas y frecuencia cardiaca en diferentes estados fisiológicos, mostrando relación entre dos sistemas corporales (neurológico y cardiovascular) que generalmente se estudian de forma aislada.

%------------------------------------------------

\section{Métodos}
Debido al corto tiempo de vida del campo en cuestión, se decide revisar todos los artículos del Dr. Plamen Ivanov que incluyan redes fisiológicas a partir del año 2012 (inclusivo).
Para ser incluídos en la revisión, los artículos deben contener un método de construcción o análisis de redes fisiológicas inédito, o ser pertinentes a una nueva patología o estado fisiológico que no hubiese sido estudiado con con anterioridad.
Se utiliza una búsqueda por autor (el Dr Plamen Ivanov) con el fin de no omitir artículos relevantes del fundador y principal publicador en el tema. Posteriormente se plantean en la búsqueda todos los campos que contengan la palabra clave "network". Al final la primera búsqueda queda planteada de la siguiente manera:

\texttt(Ivanov, Plamen Ch[Author]) AND (network) ,

Obteniendo como resultado 25 coincidencias. Se realiza una lectura diagonal de cada uno de los 25 artículos para obtener una perspectiva de los diferentes elementos históricos y técnicos que conforman redes fisiológicas.
Siendo así, 3 de los artículos cumplen con el criterio de inclusión (Tabla \ref{tab:plamenGroup}). Cabe hacer notar que la revisión sistemática del año 2016 de el grupo del Dr. Plamen\cite{ivanov2016focus} oriento a la revisioń del campo dental\cite{scala2014complex} y de Parkison\cite{monti2018network} con respecto a las redes fisiológicas.

Se plantea una segúnda búsqueda con el término "'network physiology'", que da como resultado 243 resultados.

Se lee a través del título y resumen de todos ellos para  encontrar los que son pertinentes a la fisiología de redes en seres humanos y que no hubieran sido incluídos en la búsqueda anterior. Se encontraron 22 artículos, a los cuales se les aplica la lectura en diagonal.
Al final, 7 de los artículos cumplen con los criterios de inclusion (descritos en la Tabla \ref{tab:allGroups}).

Además se presta especial atención a las aplicaciones de fisiología de redes por parte de la UNAM en el centro de estudios de la complejidad C3, que ha sido impulado por el Dr. A Barajas y la Dra. A.L. Rivera.

Una vez obtenido este conocimiento, se comienza la redacción de los resultados, dividiendo la temática en 4 subtítulos relevantes:
\begin{itemize}
  \item \textbf{Time Delay Stability (TDS)}: por la importancia que tiene en el campo, como iniciador y amplio uso dentro del campo de las redes fisiológicas.
  \item \textbf{Diferentes métodos de acoplamiento}: ya que el descubrimiento de diferentes métodos de acoplamiento simultáneos en una misma pareja de series de tiempo es extremadamente relevante.
  \item \textbf{Aplicación de teoría de redes}: para explicar las múltiples técnicas utilizados por los diversos grupos de investigación que han dedicado trabajos a las redes fisiológicas
  \item \textbf{Problemas abordados por medio de redes fisiológcias}: sección prevista para comentar la cantidad de aplicaciones dentro de los diversos campos biológicos en ciencias de la salud que pueden beneficiarse del estudio de las redes fisiológicas.
\end{itemize}

%------------------------------------------------

\section{Resultados}
La creación e interpretación de redes fisiológicas es un proceso que implica la fusión de varios elementos \cite{barajas2021sex}, que incluyen:
\begin{itemize}
  \item \textbf{Identificación y definición de los nodos que conformarán la red:} Se definen las señales que formarán parte de la red a forma de nodos.
  \item \textbf{Métodos para definir los enlaces:} El método que se va a utilizar para acoplar las señales y así determinar los enlaces de la red.
  \item \textbf{Cálculo e interpretación de propiedades topológicas y dinámicas:} \cite{adams2021gabaergic} Una vez construida la red, se deben realizar los análisis dinámicos y topográficos pertinentes e interpretar los resultados.
\end{itemize}
\onecolumn
\begin{center}
  \begin{table}[t]
    \begin{tabular}{|lp{6cm}p{6cm}|}
      \hline
      Estudio & Objetivo & Novedad en el campo \\
      \hline
      Bashan et al. (2012) \cite{bashan2012network} &
      Identificar las propiedades de las redes entre el estado de despierto y los diferentes estados de sueño.
      &
      Introduce el concepto de redes fisiológicas. Explica por primera vez el método "Time Delay Stability" en este contexto.
      \\ \hline
      Bartsch et al. (2014)\cite{bartsch2014coexisting} &
      Demostrar la necesidad de utilizar diferentes métodos de acoplamiento entre dos señales fisiológicas.
      &
      Utiliza la \textit{sincronización de fase} y los \textit{sincrogramas} en el contexto de redes fisiológicas.
      \\ \hline
      Lin et al. (2015) \cite{2015Plasticity} &
      Describir la interrelación de ondas cerebrales en diferentes regiones del cerebro.
      &
      Ahonda en los fenómenos intracerebrales del estudio publicado en el año 2012. Esta vez observamos diferentes regiones cerebrales como nodos.
      \\ \hline
    \end{tabular}
    \caption{Tabla que incluye los 4 resultados encontrados por la búsqueda "(Ivanov, Plamen Ch[Author]) AND (network) ".}
    \label{tab:plamenGroup}
  \end{table}
\end{center}
\begin{center}
  \begin{table}[t]
    \begin{tabular}{|lp{6cm}p{6cm}|}
      \hline
      Estudio & Objetivo & Novedad en el campo \\
      \hline
      Scala et al. (2014) \cite{scala2014complex}&
        Creación de una red fisiológica a través de puntos anatómicos en imágenes radiológicas haciendo énfasis en el antes y después del tratamiento de la patología.
        & Primera descripción de patología dental con el abordaje de redes fisiológicas.
      \\\hline
      Chmiel et al. (2014) \cite{chmiel2014spreading} &
      Describir la cantidad y características de las patologías del ser humano en el transcurso de la vida (diferentes grupos etarios)
      &
      La descripción de una red utilizando patologías como nodos, utilizando métodos de agrupación por medio de las características de los nodos, y luego formando enlaces entre comunidades utilizando la comorbilidad.
      \\\hline
      Monti et al. (2018) \cite{monti2018network} &
      Identificar genes y proteínas implicadas en la enfermedad de Parkinson, utilizando teoría de redes se identifican los procesos celulares posiblemente implicados. & Primera descripción de la enfermedad de Parkinson en el contexto de redes fisiológicas (proteómica y genética).
      \\ \hline
      Eastib et al. (2020) \cite{easton2020metabolic} &
      Identificar diferentes grupos etarios como estados fisiológicos basados en variables fisiológicas puntuales
      &
      Teorizar grupos etarios como estados fisiológicos.
      La utilización de variables puntuales desde un punto de vista clínico (a diferencia de la genómica o proteómica).
      Utilización de la correlación de Spearman como método para generar acoplamiento entre las variables.
      \\\hline
      Barajas-Martinez et al. (2021) \cite{barajas2021physiological}
      &
      Primer intento con el fin de describir con el mayor detalle posible la fisiología de red humana en el estado fisiológico, desde un punto de vista bioquímico. Incluyendo 54 biomarcadores.
      Utilización de métodos avanzados de detección de comunidades (algoritmos no supervisados).
      &
      Primera definición del estado fisiológico en sí mismo, basado en biomarcadores puntuales.
      \\\hline
      Barajas-Martinez et al. (2021) \cite{barajas2021sex}
      &
      Encontrar diferencias entre las redes fisiológicas formadas por biomarcadores, utilizando el sexo como estado fisiológico.
      &
      Discernir las redes fisiológicas basadas en el sexo.
      Utilizar algoritmos de teoría de redes en cuanto a la detección de comunidades.
      \\\hline
    \end{tabular}
    \caption{Tabla que incluye los 7 resultados encontrados por la búsqueda "network physiology". También incluye los del grupo de investigación C3 de la Universidad Nacional Autónoma de México}
    \label{tab:allGroups}
  \end{table}
\end{center}
\twocolumn
Para cumplir con estos tres objetivos se han utilizado diversas técnicas a través de los años, las cuales se describen a continuación.
\subsection{Time Delay Stability (TDS)}
La primera innovación propuesta por el doctor Plamen Ivanov en el año 2012 es el time delay stability {TDS} \cite{bashan2012network}, los sistemas fisiológicos funcionan a través de un conjunto de retroalimentaciones positivas y negativas, donde el aumento en una de las señales debería llevar al aumento o disminución de las señales \textit{acopladas} a ella. Es posible medir esto a través de correlación simpĺe, pero en ese caso se perdería la relación si los cambios en una señal tardaran un tiempo determinadas en verse reflejadas en la otra (aún cuando este tiempo de "desfase" fuera constante).
Para resolver este problema, se puede recurrir a la segmentación de las señales y el cálculo de su función de correlación dándonos así el desfase de una señal con respecto a otra. Pero la fuerza del método de la TDS está en tomar los valores $\tau$ de la función de correlación y compararlos en sí, creando entonces una evaluación de su \textit{estabilidad}.

Para el cálculo de la TDS, el primer paso para el cálculo de esta métrica consiste en tomar dos series de tiempo correspondientes a dos variables fisiológicas de interés $x$ y $y$ con un número de puntos $N$ que dividiremos en una número $N_L$ de segmentos $v$ parcialmente superpuestos, todos ellos de igual longitud $L$.
La superposición puede definirse como convenga, en este ejemplo usaremos $L/2$
Posteriormente se normalizará la serie de tiempo de cada segmento $v_{1...N_L}$ a media cero y desviación estándar unitaria por medio de la siguiente fórmula:
\begin{equation}
  v^\prime_t = \frac{x_t-\bar{x}}{\sigma} ,
\end{equation}
esto con el fin de poder utilizar la siguiente función de correlación sobre cada uno de los segmentos, obtenido posterior a la normalización:
\begin{equation}
  C^v_{xy}(\tau)=\frac{1}{L}\sum^L_{i=1}x^v_{i+(v-1)L/2}y^v_{i+(v-1)L/2+\tau} ,
\end{equation}
Definiremos así $\tau^v_0$ como el retraso de tiempo $\tau$ que maximice la función $C^v_{xy}$ para este segmento $v$, y definiremos una nueva serie de tiempo ${\tau^v_0}=1,...,N_L$.
Consideraremos que ambas series están acopladas en los periodos de tiempo donde los valores de la nueva serie $\tau_0$ se mantengan constantes.
Aunque el TDS proporciona una medida fidedigna de acoplamiento entre dos señales y es posible utilizarla entre dos series con diferentes unidades (por ejemplo: frecuencia cardiaca -latidos por minuto- y presión arterial -milímiteros de mercurio) e incluso diferente magnitudes (frecuencia respiratoria -en $1x10^1$ y señal electrocardiográfica - en $1x10^{-3}V$). 

Se ha mostrado que los sistemas fisiológicos pueden tener dos o más acoplamientos simultáneos y que operan a diferentes escalas.
Por lo tanto se ha propuesto utilizar diferentes métodos de acoplamiento en las señales, para capturar estos acoplamientos que pueden resultar invisibles para alguno de los métodos.

\subsection{Diferentes métodos de acoplamiento}
Si dos sistemas fisiológicos tienen dos métodos de acoplamiento que funcionan a diferentes escalas temporales (con órdenes de magnitud de diferencia) será imposible que un sólo método (TDS) pueda detectarlos.

Un ejemplo de esto es la frecuencia cardiaca, y su acoplamiento con la frecuencia respiratoria \cite{bartsch2014coexisting}.
Las señales utilizadas pueden ser el flujo de aire nasal o el perímetro torácico, incluso puede crearse una nueva señal mezclando los mejores puntos de ambas \cite{bartsch2014coexisting}, que se utilizarán para crear una onda sinusal que represente el ciclo de frecuencia respiratoria como la Figura \ref{fig:sinusoidalBreathing}.
\begin{figure}[H]
  \includegraphics[width=8cm]{sinusoidalBreathing.png}
  \caption{Se pueden observar diferentes métodos de coupling existentes entre las mismas señales. En rojo PS, en azul TDS.}
  \label{fig:sinusoidalBreathing}
\end{figure}
Uno de los procesos para evaluar acoplamiento en señales no lineales es evaluarlas a través de sincrogramas \cite{bartsch2014coexisting}.
En este caso se van a graficar los eventos de la una segunda señal no oscilatoria (en este caso electrocardiograma) sobre una señal continua oscilatoria que es la señal de frecuencia respiratoria $r(t)$. El evento de segunda señal se puede definir como el punto más alto del complejo QRS (pico de onda R) en el electrocardiograma $t_k$.
Una vez hecho esto se puede identificar la fase respiratoria instantanea $\phi_r(t)$, correspondiente a un evento de electrocardiograma determinado. Así  $\phi_r(t)$ representará el ángulo de la señal $r(t)$ y su transformada de Hilbert $r_H(t)$ que a su vez corresponde a la parte imaginaria de $\phi_r(t)$.
La gráfica de ángulos $\phi_r(t_k)$ nos otorga el sincrograma que se puede observar en la Figura \ref{fig:synchrogram}. Se define entonces el acoplamiento si existe un numero $n$ de líneas paralelas, donde $n$ es el número de latidos por ciclo respiratorio, como en la Figura \ref{fig:linePlotting}.

\begin{figure}[H]
  \includegraphics[width=8cm]{linePlotting.png}
  \caption{Grafica de 3 líneas horizontales paralelas formadas, formadas por el primer, segundo y tercer latido dentro de 3 ciclos respiratorios consecutivos. Se puede ver claramente como hay una sincronización 3:1, con 3 latidos cardiacos en cada ciclo respiratorio}
  \label{fig:linePlotting}
\end{figure}

\begin{figure}[H]
  \includegraphics[width=7cm]{synchrogram.png}
  \caption{Sincrograma, sobre la señal oscilante (señal respiratoria), se grafican puntos incidentes de electrocardiograma $t_k$ en los ángulos de fase $\phi_r(t_k)$}
  \label{fig:synchrogram}
\end{figure}

Así pues se obtienen un método de acoplamiento que puede ver sincronizaciones diferentes a las del TDS, como se puede observar en la Figura \ref{fig:coexistingFormOfCoupling} que evidencía acoplamiento en TDS (azul) diferente al sincronización de fase (PS) en (rojo).

\begin{figure}[H]
  \includegraphics[width=8cm]{coexistingFormOfCoupling.png}
  \caption{Se pueden observar diferentes métodos de coupling existentes entre las mismas señales. En rojo PS, en azul TDS.}
  \label{fig:coexistingFormOfCoupling}
\end{figure}
%------------------------------------------------

\subsection{Aplicación de teoría de redes}
Los estudios publicados en fisiología de redes han utilizado desde sus inicios análisis básicos de redes para obtener información posterior a la creación de la red fisiológica.
Desde los primeros estudios incluyendo redes fisiológicas se ha podido diferenciar estados fisiológicos por medio de métricas topográficas simples como se observa en la Figura \ref{fig:topograficProperties}

\begin{figure}[H]
  \includegraphics[width=8cm]{topograficProperties.png}
  \caption{En la figura a) se puede observar una diferencia clara en el número de enlaces que tiene la red de despierto (W) contra la de sueño profundo (DS). En la b) se puede observar una diferencia entre la fuerza promedio de los enlaces entre los mismos grupos.}
  \label{fig:topograficProperties}
\end{figure}

El equipo del Dr Plamen incorporó un análisis topográfico basado en el grado de los nodos y la fuerza de los enlaces (en unidades de \% de TDS) para diferenciar las redes fisiológicas en los estados fisiológicos de su interés \cite{bashan2012network}\cite{2015Plasticity}.
Se puede ver en la Figura \ref{fig:degreeDist} como la fuerza entre los enlaces en relación al grado de los mismos puede discernir entre estados fisiológicos, particularmente entre despierto y sueño profundo.

\begin{figure}[H]
  \includegraphics[width=8cm]{degreeDist.png}
  \caption{La fuerza de los enlaces va decreciendo conforme al grado del nodo aumenta en cada uno de los diferentes estados fisiológicos en estudio (diferentes estados de sueño y estado de despierto). La caída en el estado de sueño profundo es más pronunciada que en la de despierto.}
  \label{fig:degreeDist}
\end{figure}

Análisis más recientes han incorporado técnicas más avanzadas de redes fisiológicas, que incluyen algoritmos de detección de comunidades como Louvain y caminantes aleatorios\cite{barajas2021sex}\cite{barajas2021physiological}\cite{easton2020metabolic}\cite{chmiel2014spreading}.
Estas técnicas nos permiten observar cómo se agrupan los nodos en diferentes estados fisiológicos, y llama la atención como en ciertos estados, los mismos nodos se agrupan en diferentes comunidades, y además que las comunidades tienen diferentes enlaces que a su vez tienen diferentes características, como en la Figura\ref{fig:sexDifferences}.

\begin{figure}
  \includegraphics[width=8cm]{sexDifferences.png}
  \caption{Observar las diferencias entre las redes que representan dos estados fisiológicos visiblemente diferentes: sexo masculino a la izquierda y femenino a la derecha. El tamaño del nodo representa su eigencentralidad, el algoritmo de detección de comunidades Louvain produce las comunidades que se reperesentan en los colores de fondo. El ancho del enlace representa la correlación de Spearman entre los nodos y además podemos ver enlaces intracomunidad en negro y con rojo los enlaces entre comunidades.}
  \label{fig:sexDifferences}
\end{figure}

En otros campos relacionados, también se ha encontrado uso para las redes fisiológicas en la identificación de enfermedades dentales que afectan la estructura craneofacial\cites{scala2014complex}.
Por medio de diversos estudios de imagen, la medición con base en puntos antropométricos de interés y un análisis de redes, fue posible identificar diferencias entre los pacientes previo al tratamiento comparado con el mismo grupo después del tratamiento por medio del estudio de las redes formadas por estos datos.


\subsection{Problemas abordados por medio de redes fisiológicas}
Las redes fisiológicas prometen ofrecer una forma de abordar los sistemas biológicos de una forma integrativa, multisistémica y objetiva, uniendo diversas señales y resumiendo sus características en estados fisiológicos discernibles\cite{ivanov2016focus}.
Sin embargo aún son relativamente pocos los problemas que se han encarado por medio de ésta novedosa técnicas, en parte por ser un método de reciente creación.

El ejemplo más clásico de estados fisiológicos estudiados por esta nueva métodología es el establecido por el grupo del Dr Plamen.
Como ya se han comentado anteriormente, su enfoque principal radica en la clasificación de las redes fisiológicas separadas por etapas de sueño y del estado de despierto, en las cuales se ha aplicado un abanico diverso de análisis matemáticos, concretamente de teoría de redes a lo largo de los años\cite{bashan2012network}\cite{lin2020dynamic}.

\subsubsection{Investigación en redes fisiológicas: Centro de investigación C3 de la Universidad Nacional Autónoma de México}
La novedosa investigación del grupo de las Dra. Ana Leonor Rivera, como ya se ha mencionado con anterioridad, ha utilizado el sexo de los pacientes como un factor para dividir estados fisiológicos en el año 2021\cite{barajas2021sex}.

Dentro de los abordajes de éste grupo también encontramos el enfoque, diferente al del grupo del Dr. Plamen, con la integración de variables fisiológicas tomadas en un punto en el tiempo (ejemplo: marcadores hepáticos, renales, perfil de lípidos, frecuencia respiratoria...) en contraste con las series de tiempo que se venían manejando, para Posteriormente utilizar estas variables como nodos en una red.
Es así como se logran utilizar biomarcadores fisiológicos que indican presencia de enfermedades de gran importancia en diferentes sistemas, incluyendo el sistema endocrino, hepático, pulmonar entre otros \cite{barajas2021physiological}, representandolos como nodos en una red fisiológica, mostrando así una representación del estado fisiológico del paciente en un punto estático del tiempo Figura \ref{fig:topograficProperties}

\section{Discusión}
El nuevo campo de las redes fisiológicas proporciona un entendimiento novedoso e integral de los sistemas en diversos campos de la ciencia.
La inclusión de teoría de redes en los primeros estudios del Dr. Plamen es relativamente pobre si lo comparamos con las técnicas de análisis que se han venido implementando en los últimos años.
Tanto su grupo de investigación como el resto de individuos involucrados en el campo han ido incorporando cada vez más rigurosos análisis de redes.

Aunque se continúan encontrando aplicaciones para las redes fisiológicas dentro del campo de la investigación médica y se logran discernir estados fisiológicos desde las diferentes perspectivas ya mencionadas, es importante continuar la búsqueda de nuevos enfoques, incorporar cada vez más elementos de la ciencia de redes (que ya ha aportado bastante al campo). 
Prácticamente todas las disciplinas médicas pueden beneficiarse de esta nueva perspectiva, en especial en aquellas enfermedades altamente prevalentes que tienen los más altos índices de morbimortalidad a nivel mundial, como la hipertensión arterial y sus complicaciones o enfermedades renales.

Es claro que conseguir la cantidad de datos necesarios para plantear el estudio de redes en cualquier patología es un reto para los investigadores, sin embargo el avance tecnológico que nos permite capturar series de tiempo con diversos sensores cada vez más sofistiacados ofrece soluciones nuevas cada día para solventar ésta problemática.

%----------------------------------------------------------------------------------------
%	REFERENCE LIST
%----------------------------------------------------------------------------------------
\renewcommand\refname{Referencias}
\printbibliography
\end{document}
